% vim: set tw=78 sts=2 sw=2 ts=8 aw et ai:

There are three areas which have remained unattained, and with good reason, because of their complexity.
I plan on supporting these some time in the future.

\begin{enumerate}

\item{Different Architectures}

The current implementation is tied to the x86 architecture.
Current virtualization technologies are unfortunately very restrictive (It's hard to emulate PPC instruction set on a x86_64 bare metal).
This is why support for these kind of features in the industry is limited.
However, QEMU is a CPU simulator that is gaining more and more ground, and would be a good candidate for such a task.

\item{Different Language versions}

What is an user wants to compile a program using Python 2.6, or Python 3.0?. The same stands with C++98 vs C++11.
There is more than one standard for each language.
In order to provide support for this option, one would need a program that can change the runtime execution context.
Another option would be to have different VM's, each with its own run environment. 

\item{Kernel code}

Compiling and running kernel code is difficult, because once in the kernel one typically has control over the whole system.
Needless to say, we cannot simply compile kernel modules that users send, and insert them on our machine.
A possible solution to this problem would involve using User Mode Linux, which is Linux run inside a regular process, providing the regular encapsulation that processes have, and would allow us to hook up with the existing ptrace mechanism, that is already implemented.
\end{enumerate}


Writing SourceDotCode has been a challenging task. Although it still lacks some features, it's an useful tool to have around, when a compiler is missing.
