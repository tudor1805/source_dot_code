% vim: set tw=78 sts=2 sw=2 ts=8 aw et ai:

We define Compile as a Service (CAAS), as a function taking as input a superset of the following parameters:
\begin{enumerate}
\item{Source code}
\item{Language}
\item{Architecture}
\end{enumerate}
and providing as output the execution result of running the program.


As an inspiration for this project I had two platforms: 
\begin{enumerate}
\item{Infoarena}
\item{VMChecker}
\end{enumerate}

They are both great platforms, but they are rigid, in the sense that they are targeted to certain operations.
They are not general purpose compilation platforms, which forms the scope of my project.

Infoarena only allows submitting code for certain programming challenges.
As supported languages, it only allows submitting C and C++ code, differentiating between them by the file extension.

VMChecker, is the platform for evaluating homeworks for a good amount of subjects, inside the Faculty of Automatic Control and Computers.
It is a bit more complex than Infoarena, and it allows compiling programs in a greater amount of programming languages.
Programs are typically compiled and run inside a Virtual Machine environment, hence the name "VMChecker".


When designing an online compilation tool, there are certain challenges, most notably:
\begin{enumerate}
\item{Scalability}
\item{Security}
\end{enumerate}

Scalability refers to how the application is going to handle a large number of incoming requests.
Common bottlenecks in such applications occur between the Web Server and Database, and between the Web Server and secodary resources (Ex: When calling another Web Service for a specific task).

The security issues that may arise in our case deal with the code execution.
Allowing the user to execute arbitrary code on your machine would pose a huge risk.
For example, he could attempt to reboot the machine, or create sockets and flood a destination with ICMP packets.

In the next section, we are going to explore a way to overcome this problem.










 
